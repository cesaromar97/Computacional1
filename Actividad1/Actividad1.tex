\documentclass[12pt]{article}
\usepackage[a4paper]{geometry}
%\userpackage[top=1 in, bottom=1.25 in, left=1.1 in, rigth=1.1 in] {geometry}
%\usepackage[paperwidth=17cm, paperheight=22.5cm, bottom=2.5cm, right=2.5cm]{geometry}
\usepackage[utf8]{inputenc}
%\usepackage[a4paper, top=2.5cm, bottom=2.5cm, left=2.2cm, right=2.2cm]
%{geometry}
%\usepackage[myheadings]{fullpage}
\usepackage{fancyhdr}
\usepackage{lastpage}
%\usepackage{float}
\usepackage{graphicx, wrapfig, subcaption, setspace, booktabs}
\usepackage{graphicx}
\usepackage[T1]{fontenc}
\usepackage[font=small, labelfont=bf]{caption}
%\usepackage{fourier}
\usepackage[protrusion=true, expansion=true]{microtype}
\usepackage[english]{babel}
\usepackage{sectsty}
\usepackage{url, lipsum}
\usepackage[T1]{fontenc}
\usepackage{icomma}
\usepackage{siunitx}
\usepackage{ragged2e}
\usepackage{amsmath}
\usepackage{comment}
\usepackage{enumerate}
%\usepackage{changepage}
\usepackage{anysize}




\newcommand{\HRule}[1]{\rule{\linewidth}{#1}}
\onehalfspacing
\setcounter{tocdepth}{5}
\setcounter{secnumdepth}{5}

%-------------------------------------------------------------------------------
% HEADER & FOOTER
%-------------------------------------------------------------------------------


\begin{comment}
-Udledninger
$$
\begin{aligned}


\end{aligned}
$$

-Opgavetekst
\begin{figure}[H]
\includegraphics[width=0.5\textwidth]{"path"}
\end{figure} 


-Opgave billede med tekst
\begin{figure}[H]
\caption{"Billedtekst"}
\includegraphics[width=0.5\textwidth]{"path"}
\end{figure} 

-Værdier
$\\

$


\end{comment}
\begin{document}

\begin{titlepage}

\title{ \normalsize 
		%\begin{figure}
        \begin{center}
        \includegraphics[height=6cm]{Logo.jpg}
        \end{center}
       % \end{figure}
        \LARGE \textsc{\textbf{Universidad De Sonora}} \\ \bigskip
		\Large División de Ciencias Exactas y Naturales \\
        Licenciatura En Física \\ \bigskip
        \bigskip
        Física Computacional I
		\\ [0.1cm]  
		\HRule{2pt} \\
		\Large \textbf{{Reporte de Actividad 1}} \\
        \textit{\textbf{"Atmósfera Terrestre"}}
		\HRule{2pt} \\
		\normalsize \vspace*{0.001\baselineskip}}
        
\date{\bigskip \Large Hermosillo, Sonora  \hspace*{\fill}  Enero 30 de 2018}

        
\author{
		\Large\textbf{ César Omar Ramírez Álvarez} \\ \bigskip
        \\ \bigskip
       \Large Profr. Carlos Lizárraga Celaya}
       \end{titlepage}
       \maketitle
       

\newpage
\pagestyle{plain}
\section{Introducción}

\section{Composición}
	Los tres componentes principales del aire, y por lo tanto de la atmósfera de la Tierra, son nitrógeno, oxígeno y argón. El vapor de agua representa aproximadamente el 0.25\% de la atmósfera en masa. Los gases restantes a menudo se denominan gases traza, entre los cuales se encuentran los gases de efecto invernadero; principalmente dióxido de carbono, metano, óxido nitroso y ozono. El aire filtrado incluye trazas de muchos otros compuestos químicos. Muchas sustancias de origen natural pueden estar presentes en pequeñas cantidades localmente y estacionalmente variables como aerosoles en una muestra de aire sin filtrar, que incluye polvo de minerales y composición orgánica, polen y esporas, rocío de mar y cenizas volcánicas. Varios contaminantes industriales también pueden estar presentes en forma de gases o aerosoles, como cloro (elemental o en compuestos), compuestos de flúor y mercurio elemental o vapor. Compuestos de azufre que se pueden derivar de fuentes naturales o de la contaminación del aire industrial.
\section{Estructura De La Atmósfera}

\subsection{Capas Principales}
En general, la presión del aire y la densidad disminuyen con la altitud en la atmósfera. Sin embargo, la temperatura tiene un perfil más complicado con la altitud, y puede permanecer relativamente constante o incluso aumentar con la altitud en algunas regiones. 
\subsubsection{Exósfera}

\subsubsection{Termósfera}

\subsubsection{Mesósfera}

\subsubsection{Estratósfera}

\subsubsection{Tropósfera}

\subsection{Otras Capas}

\section{Propiedades Físicas}

\subsection{Presión Y Espesor}

\subsection{Temperatura Y Velocidad Del Sonido}

\subsection{Densidad Y Masa}

\section{Propiedades Ópticas}

\subsection{Dispersión}

\subsection{Absorción}

\subsection{Emisión}

\subsection{Índice De Refracción}

\section{Circulación}

\end{document}